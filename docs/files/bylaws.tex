\documentclass{article}
\usepackage[UTF8]{ctex}
\usepackage{amsmath}
\usepackage{amssymb}
\usepackage{geometry}
\usepackage{zhnumber}
\usepackage{enumerate}
\usepackage{enumitem}
\usepackage{multicol}
\usepackage{makecell}
\renewcommand\thesection{第\chinese{section}章}

\geometry{a4paper, scale=0.72}

\title{镇海中学学生义工管理细则}
\author{镇海中学团委、学生会实践部}
\date{2023 年 10 月}

\begin{document}
\maketitle

\section{总则}

\begin{enumerate}[label=\heiti{\textbf{第\chinese*条}}, leftmargin=4em]
  \fangsong
  \item 为了培育和践行社会主义核心价值观,促进劳动教育,保障和规范学生义工活动的有效开展,制定本细则。
  \item 镇海中学学生义工是由学校团委统筹、学生会组织、全体同学共同参与的公益性志愿服务活动,由学生会实践部负责安排与管理,义工管理委员会负责配合做好落实、动员等工作。
  \item 学生义工的完成情况作为我校学生参与劳动教育、志愿服务、社会实践的集中体现,将以一定形式载入浙江省普通高中学生综合素质评价系统。
\end{enumerate}

\section{义工活动}

\begin{enumerate}[resume, label=\heiti{\textbf{第\chinese*条}}, leftmargin=4em]
  \fangsong
  \item 学生义工活动分为校内义工、校外义工与大型社会实践义工三类。\\
  校内义工是学生参加学校组织的校园活动服务、劳动卫生、学生工作等活动;\\
  校外义工是学生利用周末、小长假等课余时间在博物馆、体育馆、展览馆、医院、养老院、福利院、社区、农村、工厂等社会服务场所进行的志愿服务活动;\\
  大型社会实践义工是学生在寒暑假期参加学校组织校外社会实践活动。
  \item 校内义工由学生会实践部负责组织管理。在校园活动需要时,学生会各部门可以向全校征集义工,\textbf{具体内容应向学生会实践部提前报备},学生会成员未经报备,不得向全校征集义工。\textbf{义工时间发放时长应由实践部视具体情况核准},\textbf{否则一律按最低标准发放}。
  \item 校外义工由各班团支书组织管理,原则上要求 $4$ 至 $8$ 人为单位组队完成。活动中各组要保质保量完成工作并进行整理报告。团支书应在每次活动前做好动员组织、分配队伍、数据统计等工作,在活动后收集、汇总各组报告,确认材料属实,并统一收集好以照片形式上传至义工管理平台。
  \item 大型社会实践义工由学生会实践部组织,各班同学参与,原则上每次活动每位同学应至少参加一项。
  \item 需要参加义工的需以班级或社团为单位上报人员名单至平台,并由一位带队同学带领。
\end{enumerate}

\section{义工管理平台与义工时间管理}

\begin{enumerate}[resume, label=\heiti{\textbf{第\chinese*条}}, leftmargin=4em]
  \fangsong
  \item 义工管理平台是在校期间义工活动的统计平台,包括义工发布、义工报名、感想审核、义工数据统计等功能。在校期间,所有义工管理事务均在义工管理平台上完成。义工管理平台由学生会实践部领导,义工自主管理委员会和各班团支书共同管理。
  \item 所有校内义工任务及其时间、地点、内容等信息均会在义工管理平台上呈现,团支书应在义工任务前按义工要求完成报名。如有临时任务的,团支书应统计好参加义工的人员名单并在义工结束后的一天内及时在平台上提交相关信息。
  \item 校外义工由团支书在义工管理平台上完成报名,义工证明材料上传至义工管理平台,由学生会实践部和义工自主管理委员会审核。
  \item 大型社会实践义工由各班团支书将本班社会实践资料打包上传至 FTP 文件夹,由学生会实践部进行审核。
  \item 校内义工和校外义工结束后参与义工的学生需在义工结束后五天内在义工管理平台上提交义工感想,感想将由学生会实践部和义工自主管理委员会审核。对于感想敷衍潦草不合格的,将打回重新提交。对于多次出现此类情况的将适当减少义工时间。
  \item 参与义工活动的时间由学生会实践部认定。校内义工视情况认定时长。校外义工时间原则上一次不得超过5小时。大型社会实践义工时间一般为每次6小时,认定为优秀的可以适当增加,认定不合格的,发放 $3$ 小时。
  \item 每位同学在毕业前至少需要完成 $30$ 小时的校内义工、$15$ 小时的校外义工和 $18$ 小时的大型社会实践义工,纳入个人高中综合素质评价。义工总时长不达标的,一定程度上影响评优评先、三位一体和强基招生等推荐资格。
  \item 溢出机制。若校内义工时间已满,超出的部分按 $\dfrac13$ 折算进校外义工时间,总的溢出时长不超过 $6$ 小时(四舍五入)。
  \item 个人获奖类义工总时长不超过 $10$ 小时,可由个人选择计入校内义工时长或者校外义工时长。若以集体为单位参赛的,视具体情况分层发放。
  \item 任何人不得通过造假、冒名顶替等手段获取义工时间,各班团支书应做好监督检查工作,认真审核。\\
  违规行为按情节严重程度,经团委核准,处以取消当次义工时间、倒扣义工时间等处罚。违反校纪校规的,按照相关规定给予相应处分。
  \item 所有因为团支书疏忽造成的义工时间无法正常发放情况,包括但不限于未及时报名或义工感想未及时提交,后果由团支书自行负责。
  \item 以社团(包括合唱队等校级队伍)为单位提交义工时间申请的,需提前统一上交一份分层发放义工时间的成员名单,并有负责老师和同学签名,交至实践部。义工时间发放遵照 ``常见义工活动时间核准参照表'' 执行,若确有调整,需提前与负责老师确认,并报实践部核准。义工时间结算需在十月前完成,逾期视作无效。分层发放依据是出勤状况、贡献程度、活动态度等,具体由活动负责老师和同学斟酌确定。
  \item 对于校内义工和校外义工,完成优秀的,适当增加义工时间;完成不合格的,考虑适当减少义工时间发放。
\end{enumerate}

\section{附则}

\begin{enumerate}[resume, label=\heiti{\textbf{第\chinese*条}}, leftmargin=4em]
  \fangsong
  \item 本细则由学生会实践部负责解释,细则内容变更由团委和实践部讨论决定。
  \item 本细则自2023年9月起施行。
  \item 本细则从2023级开始实行。
\end{enumerate}

\begin{center}
  \flushright
  \songti
  镇海中学团委、学生会实践部\\
  2023 年 9 月
\end{center}

\appendix

\section{常见义工活动时间核准参照表}

\renewcommand\thesubsection{\chinese{subsection}}

\subsection{体力劳动}

\subsubsection*{整理打扫类}

\begin{tabular}{cc}
  \hline
  \textbf{类型} & \textbf{时间} \\
  \hline
  图书馆 & $1 \sim 2$ 小时 \\
  体艺馆 & $1 \sim 3$ 小时 \\
  音乐教室 & $1 \sim 2$ 小时 \\
  天文台 & $3$ 小时 \\
  选修教室 & $1$ 小时 \\
  其他公共区域 & $2$ 小时以下 \\
  \hline
\end{tabular}

\subsubsection*{校内事务类}

\begin{tabular}{cc}
  \hline
  \textbf{类型} & \textbf{时间} \\
  \hline
  各种节日活动布置 & $1 \sim 3$ 小时 \\
  假期护校 & 每半天 $4$ 小时 \\
  \hline
\end{tabular}

\subsubsection*{大型考试类}

\begin{tabular}{cc}
  \hline
  \textbf{类型} & \textbf{时间} \\
  \hline
  布置体育馆 & $2$ 小时 \\
  引导员、志愿者 & $1 \sim 3$ 小时 \\
  学选考义工、高考义工 & 每半天 $5$ 小时 \\
  \hline
\end{tabular}

\subsection{脑力劳动}

\begin{tabular}{cc}
  \hline
  \textbf{类型} & \textbf{时间} \\
  \hline
  学校图画绘制 & $1 \sim 2$ 小时 \\
  校内活动的策划 & $2 \sim 3$ 小时 \\
  海报、手抄报 & $2 \sim 4$ 小时 \\
  征稿 & 最多 $2$ 小时 \\
  \hline
\end{tabular}

\subsection{临时任务}

\begin{tabular}{cc}
  \hline
  \textbf{类型} & \textbf{时间} \\
  \hline
  分发通知 & $1$ 小时以下 \\
  学校临时任务安排 & $1 \sim 3$ 小时 \\
  \hline
\end{tabular}

\subsection{学生活动}

\subsubsection*{校内活动}

\begin{tabular}{cc}
  \hline
  \textbf{类型} & \textbf{时间} \\
  \hline
  \textbf{服务类:}学生会部长级、\\国旗班 & 校内全满(中途退出减一半以上);学生会干事半满 \\
  \hline
  \textbf{社团类:}灵通记者团、\\数媒镇中 & 时间采取分层发放,封顶校内全满,按稿件计算 \\
  \hline
  \textbf{艺术类:}合唱团、管弦乐队、\\健美操队、舞蹈队 & 时间采取分层发放,封顶校内全满 \\
  \hline
\end{tabular}

\subsubsection*{校外活动}

\begin{tabular}{cc}
  \hline
  \textbf{类型} & \textbf{时间} \\
  \hline
  情系母校 & 时间采取分层发放,封顶校外全满 \\
  \hline
  领导力项目组、\\模拟政协比赛、\\科技新苗 & 校外全满(中途退出减一半以上) \\
  \hline
  社团特殊校外活动 & $3 \sim 4$ 小时 \\
  \hline
\end{tabular}

\subsubsection*{社团活动对学校有益的}

$1 \sim 2$ 小时

\paragraph{注:}

\begin{enumerate}
  \item 表中第四项学生组织所获奖项,不计入获奖义工
  \item 实际获得义工时长根据实际劳动时间和强度由实践部决定,本表仅供参考。
  \item 同类型校内活动只取时间最长项计入,不得叠加。
\end{enumerate}

\section{获奖义工}

\subsection{艺术类}

\begin{tabular}{cc}
  \hline
  \textbf{获奖等级} & \textbf{时间} \\
  \hline
  区级二、三等奖 & $1$ 小时 (个人) 或 $\dfrac12$ 小时 (团体) \\
  区级一等奖、市级二三等奖 & $2$ 小时 (个人) 或 $1$ 小时 (团体) \\
  市级一等奖以上 & $3$ 小时 (个人) 或 $2$ 小时 (团体) \\
  \hline
\end{tabular}

\subsection{学科类}

\begin{tabular}{cc}
  \hline
  \textbf{获奖等级} & \textbf{时间} \\
  \hline
  区级二、三等奖 & $1$ 小时 \\
  区级一等奖、市级二三等奖 & $2$ 小时 \\
  市级一等奖以上、省级二三等奖 & $3$ 小时 \\
  省级一等奖或国家级奖项 & $4$ 小时 \\
  \hline
\end{tabular}

\subsection{体育类}

\begin{tabular}{cc}
  \hline
  \textbf{获奖等级} & \textbf{时间} \\
  \hline
  区 $4\sim8$ 名 & $1$ 小时 \\
  区 $1\sim3$ 名、市 $4\sim8$ 名 & $2$ 小时 \\
  市 $1\sim3$ 名、省 $4\sim10$ 名 & $3$ 小时 \\
  省 $1\sim3$ 名、国家级名次 & $4$ 小时 \\
  \hline
\end{tabular}

\paragraph{注:}

具体情况由实践部讨论决定。
\end{document}
